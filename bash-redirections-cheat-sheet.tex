% Bash Redirections Cheat Sheet
%
% by Peteris Krumins (peter@catonmat.net)
% http://www.catonmat.net  -  good coders code, great coders reuse
%
% 2012.09.09
%

\documentclass[9pt]{memoir}

\usepackage[left=1cm,top=0.5cm,right=1cm,bottom=0.5cm,nohead,nofoot]{geometry}

\usepackage[pdftex]{hyperref}
\usepackage{verbatim}
\usepackage{array}
\usepackage{upquote}

\hypersetup{pdftitle={Bash Redirections Cheat Sheet}}
\hypersetup{pdfauthor={Peteris Krumins (peter@catonmat.net)}}
\hypersetup{pdfkeywords={cheat sheet, bash, shell, bourne again shell, sh, redirection, redirections, command line, unix, linux, cheat sheat, cheet sheet, cheet sheat}}
\hypersetup{pdfsubject={http://www.catonmat.net - good coders code, great coders reuse}}
\hypersetup{colorlinks}

\pagestyle{empty}

% -----------------------------------------------------------------------

\begin{document}

\begin{center}
\huge \textbf{Bash Redirections Cheat Sheet}
\end{center}

\vspace{0.1in}

\renewcommand{\arraystretch}{1.2}
\begin{tabular}{|m{5.5cm}|m{12.2cm}|}
\hline
\large\textbf{Redirection} & \large\textbf{Description} \\
\hline
\verb|cmd > file| & Redirect the standard output (stdout) of \verb|cmd| to a file. \\
\hline
\verb|cmd 1> file| & Same as \verb|cmd > file|. \verb|1| is the default file descriptor (fd) for stdout. \\
\hline
\verb|cmd 2> file| & Redirect the standard error (stderr) of \verb|cmd| to a file. \verb|2| is the default fd for stderr. \\
\hline
\verb|cmd >> file| & Append stdout of \verb|cmd| to a file. \\
\hline
\verb|cmd 2>> file| & Append stderr of \verb|cmd| to a file. \\
\hline
\verb|cmd &> file| & Redirect stdout and stderr of \verb|cmd| to a file. \\
\hline
\verb|cmd > file 2>&1| & Another way to redirect both stdout and stderr of \verb|cmd| to a file. This \underline{is not} the same as \verb|cmd 2>&1 > file|. \underline{Redirection order matters!} \\
\hline
\verb|cmd > /dev/null| & Discard stdout of \verb|cmd|. \\
\hline
\verb|cmd 2> /dev/null| & Discard stderr of \verb|cmd|. \\
\hline
\verb|cmd &> /dev/null| & Discard stdout and stderr of \verb|cmd|. \\
\hline
\verb|cmd < file| & Redirect the contents of the file to the standard input (stdin) of \verb|cmd|. \\
\hline
\small{
\verb|cmd << EOL| \par
\verb|line1| \par
\verb|line2| \par
\verb|EOL|
} & Redirect a bunch of lines to the stdin. If \verb|'EOL'| is quoted, text is treated literally. This is called a here-document. \\
\hline
\small{
\verb|cmd <<- EOL| \par
\verb|<tab>foo| \par
\verb|<tab><tab>bar| \par
\verb|EOL|
} & Redirect a bunch of lines to the stdin and strip the leading tabs. \\
\hline
\verb|cmd <<< "string"| & Redirect a single line of text to the stdin of \verb|cmd|. This is called a here-string. \\
\hline
\verb|exec 2> file| & Redirect stderr of all commands to a file forever. \\
\hline
\verb|exec 3< file| & Open a file for reading using a custom file descriptor. \\
\hline
\verb|exec 3> file| & Open a file for writing using a custom file descriptor. \\
\hline
\verb|exec 3<> file| & Open a file for reading and writing using a custom file descriptor. \\
\hline
\verb|exec 3>&-| & Close a file descriptor. \\
\hline
\verb|exec 4>&3| & Make file descriptor \verb|4| to be a copy of file descriptor \verb|3|. (Copy fd \verb|3| to \verb|4|.) \\
\hline
\verb|exec 4>&3-| & Copy file descriptor \verb|3| to \verb|4| and close file descriptor \verb|3|. \\
\hline
\verb|echo "foo" >&3| & Write to a custom file descriptor. \\
\hline
\verb|cat <&3| & Read from a custom file descriptor. \\
\hline
\verb|(cmd1; cmd2) > file| & Redirect stdout from multiple commands to a file (using a sub-shell). \\
\hline
\verb|{ cmd1; cmd2; } > file| & Redirect stdout from multiple commands to a file (faster; not using a sub-shell). \\
\hline
\verb|exec 3<> /dev/tcp/host/port| & Open a TCP connection to \verb|host:port|. (This is a bash feature, not Linux feature). \\
\hline
\verb|exec 3<> /dev/udp/host/port| & Open a UDP connection to \verb|host:port|. (This is a bash feature, not Linux feature). \\
\hline
\verb|cmd <(cmd1)| & Redirect stdout of \verb|cmd1| to an anonymous fifo, then pass the fifo to \verb|cmd| as an argument. Useful when \verb|cmd| doesn't read from stdin directly. \\
\hline
\verb|cmd < <(cmd1)| & Redirect stdout of \verb|cmd1| to an anonymous fifo, then redirect the fifo to stdin of \verb|cmd|. Best example: \verb\diff <(find /path1 | sort) <(find /path2 | sort)\. \\
\hline
\verb|cmd <(cmd1) <(cmd2) | & Redirect stdout of \verb|cmd1| and \verb|cmd2| to two anonymous fifos, then pass both fifos as arguments to \verb|cmd|. \\
\hline
\verb/cmd1 >(cmd2)/ & Run \verb|cmd2| with its stdin connected to an anonymous fifo, and pass the filename of the pipe as an argument to \verb|cmd1|. \\
\hline
\verb/cmd1 > >(cmd2)/ & Run \verb|cmd2| with its stdin connected to an anonymous fifo, then redirect stdout of \verb|cmd| to this anonymous pipe. \\
\hline
\verb/cmd1 | cmd2/ & Redirect stdout of \verb|cmd1| to stdin of \verb|cmd2|. Pro-tip: This is the same as \verb|cmd1 > >(cmd2)|, same as \verb|cmd2 < <(cmd1)|, same as \verb|> >(cmd2) cmd1|, same as \verb|< <(cmd1) cmd2|. \\
\hline
\verb/cmd1 |& cmd2/ & Redirect stdout and stderr of \verb|cmd1| to stdin of \verb|cmd2| (bash 4.0+ only). Use \verb/cmd1 2>&1 | cmd2/ for older bashes. \\
\hline
\verb/cmd | tee file/ & Redirect stdout of \verb|cmd| to a file and print it to screen. \\
\hline
\verb|exec {filew}> file| & Open a file for writing using a named file descriptor called \verb|{filew}| (bash 4.1+). \\
\hline
\verb|cmd 3>&1 1>&2 2>&3| & Swap stdout and stderr of \verb|cmd|. \\
\hline
\verb|cmd > >(cmd1) 2> >(cmd2)| & Send stdout of \verb|cmd| to \verb|cmd1| and stderr of \verb|cmd| to \verb|cmd2|. \\
\hline
\verb|cmd <(cmd1) <(cmd2)| & Send stdouts of \verb|cmd1| and \verb|cmd2| to descriptors, giving descriptor names to \verb|cmd|. \\
\hline
\verb/cmd1 | cmd2 | cmd3 | cmd4/ \par
\verb/echo ${PIPESTATUS[@]}/ & Find out the exit codes of all piped commands. \\
\hline
\end{tabular}

\vfill

I explained each one of these redirections in my article \href{http://www.catonmat.net/blog/bash-one-liners-explained-part-three/}{All About Bash Redirections}: \par
\href{http://www.catonmat.net/blog/bash-one-liners-explained-part-three/}{www.catonmat.net/blog/bash-one-liners-explained-part-three/}

\vfill

Did I miss any redirections? Let me know! Email me peter@catonmat.net, or fork this cheat sheet on github: \par \href{http://github.com/pkrumins/bash-redirections-cheat-sheet}{www.github.com/pkrumins/bash-redirections-cheat-sheet}

\vfill

\framebox{\parbox{5in}{
A cheat sheet by \textbf{Peteris Krumins} (peter@catonmat.net), September 2012.

\href{http://www.catonmat.net}{http://www.catonmat.net} - good coders code, great coders reuse

\vspace{2mm}
\footnotesize{Released under GNU Free Document License.}}}

\end{document}

